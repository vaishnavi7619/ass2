% \iffalse
\let\negmedspace\undefined
\let\negthickspace\undefined
\documentclass[journal,12pt,twocolumn]{IEEEtran}
\usepackage{cite}
\usepackage{amsmath,amssymb,amsfonts,amsthm}
\usepackage{algorithmic}
\usepackage{graphicx}
\usepackage{textcomp}
\usepackage{xcolor}
\usepackage{txfonts}
\usepackage{listings}
\usepackage{enumitem}
\usepackage{mathtools}
\usepackage{gensymb}
\usepackage{comment}
\usepackage[breaklinks=true]{hyperref}
\usepackage{tkz-euclide} 
\usepackage{listings}
\usepackage{gvv}                                        
\def\inputGnumericTable{}                                 
\usepackage[latin1]{inputenc}                                
\usepackage{color}                                            
\usepackage{array}                                            
\usepackage{longtable}                                       
\usepackage{calc}                                             
\usepackage{multirow}                                         
\usepackage{hhline}                                           
\usepackage{ifthen}                                           
\usepackage{lscape}

\newtheorem{theorem}{Theorem}[section]
\newtheorem{problem}{Problem}
\newtheorem{proposition}{Proposition}[section]
\newtheorem{lemma}{Lemma}[section]
\newtheorem{corollary}[theorem]{Corollary}
\newtheorem{example}{Example}[section]
\newtheorem{definition}[problem]{Definition}
\newcommand{\BEQA}{\begin{eqnarray}}
\newcommand{\EEQA}{\end{eqnarray}}
\newcommand{\define}{\stackrel{\triangle}{=}}
\theoremstyle{remark}
\newtheorem{rem}{Remark}
\begin{document}

\bibliographystyle{IEEEtran}
\vspace{3cm}

\title{NCERT 11.9.2  Q9}
\author{EE23BTECH11014- Devarakonda Guna vaishnavi $^{}$% <-this % stops a space
}
\maketitle
\newpage
\bigskip

\renewcommand{\thefigure}{\theenumi}
\renewcommand{\thetable}{\theenumi}

\bibliographystyle{IEEEtran}



\textbf{Question:}
\documentclass{}
\usepackage{}



\textbf{Question:}
The sum of the first $n$ terms of two arithmetic progressions (AP) is in the ratio $5n+4 : 9n+6$. Find the ratio of their 18th terms.
\documentclass{article}
\usepackage{amsmath}

\begin{document}

\textbf{Given Information:}

There are two arithmetic progressions (AP) with different first terms and common differences.

\textbf{For the First AP:}

\documentclass{article}
\usepackage{amsmath}

\begin{document}

\textbf{Given Information:}

There are two arithmetic progressions (AP) with different first terms and common differences.

\textbf{For the First AP:}

Let the first term be \(a\) and the common difference be \(d\). The sum of \(n\) terms is given by:
\[ S_n = \frac{n}{2}[2a + (n-1)d] \]

The \(n\)-th term is given by:
\[ a_n = a + (n-1)d \]

\textbf{For the Second AP:}

Let the first term be \(A\) and the common difference be \(D\). The sum of \(n\) terms is given by:
\[ S_n = \frac{n}{2}[2A + (n-1)D] \]

The \(n\)-th term is given by:
\[ A_n = A + (n-1)D \]

Given that the sum of \(n\) terms of the first AP to the sum of \(n\) terms of the second AP is in the ratio \(5n+4 : 9n+6\), we have the equation:
\[ a + \left(\frac{n-1}{2}\right)d \div A + \left(\frac{n-1}{2}\right)D = \frac{5n+4}{9n+6} \]

\textbf{Introducing Z-transforms:}

Let \(X(z)\) be the Z-transform of the sequence \(x_n\):
\[ X(z) = \sum_{n=0}^{\infty} x_n z^{-n} \]

Apply Z-transform to the equations for \(a_n\) and \(A_n\):
\[ A(z) = a + zd \]
\[ B(z) = A + zD \]

Now, express the given ratio in terms of Z-transforms:
\[ \frac{A(z)}{B(z)} = \frac{5z + 4}{9z + 6} \]

\textbf{To Find the Ratio of their 18th Terms in Z-transforms:}

The Z-transform of the \(18\)th term of the first AP is \(a + 17d\), and the Z-transform of the \(18\)th term of the second AP is \(A + 17D\). The ratio is:
\[ \text{Ratio} = \frac{a + 17d}{A + 17D} \]

\textbf{Solving for \(n\):}

Since \(\frac{n-1}{2} = 17\), solving for \(n\) gives \(n = 35\).

Substituting \(n = 35\) into the equation, we get:
\[ a + \left(\frac{35-1}{2}\right)d \cdot A + \left(\frac{35-1}{2}\right)D = \frac{5(35)+4}{9(35)+6} \]

Solving the equation yields \(a + 17d \div A + 17D = \frac{179}{321}\).

Hence, the ratio of the \(18\)th term of the first AP to the \(18\)th term of the second AP is \(179:321\).

\end{document}


\end{document}
